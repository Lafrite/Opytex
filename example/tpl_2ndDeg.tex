\documentclass[a4paper,10pt]{article}
\RequirePackage[utf8x]{inputenc}
\RequirePackage[francais]{babel}
\RequirePackage{amssymb}
\RequirePackage{amsmath}
\RequirePackage{amsfonts}
\RequirePackage{subfig}
\RequirePackage{graphicx}
\RequirePackage{color}

% Title Page
\title{Calcul littéral et statistiques}
\date{\today}

\begin{document}
\maketitle


\section{Polynômes}



    \Block{set P = Polynom_deg2.random(["{a}", "{b}", "{c}"], ["{b}**2 - 4*{a}*{c} == 0"])}
    Résoudre l'équation suivante
    \begin{eqnarray*}
        \Var{P} & = & 0
    \end{eqnarray*}

    Solution:

    On commence par calculer le discriminant de $P(x) = \Var{P}$.
    \begin{eqnarray*}
        \Delta & = & b^2-4ac \\
        \Var{P.delta.explain()|calculus(name="\\Delta")}
    \end{eqnarray*}

    \Block{if P.delta > 0}
    comme $\Delta = \Var{P.delta} > 0$ donc $P$ a deux racines

    \begin{eqnarray*}
        x_1 & = & \frac{-b - \sqrt{\Delta}}{2a} =  \frac{\Var{-P.b} - \sqrt{\Var{P.delta}}}{2 \times \Var{P.a}} = \Var{P.roots()[0] } \\
        x_2 & = & \frac{-b + \sqrt{\Delta}}{2a} =  \frac{\Var{-P.b} + \sqrt{\Var{P.delta}}}{2 \times \Var{P.a}} = \Var{P.roots()[1] }
    \end{eqnarray*}

    Les solutions de l'équation $\Var{P} = 0$ sont donc $\mathcal{S} = \left\{ \Var{min(P.roots())}; \Var{max(P.roots())} \right\}$

    \Block{elif P.delta == 0}
    Comme $\Delta = 0$ donc $P$ a deux racines

    \begin{eqnarray*}
        x_1 = \frac{-b}{2a} = \Var{P.roots()[0]} \\
    \end{eqnarray*}

    La solution de $\Var{P} = 0$ est donc $\mathcal{S} = \left\{ \Var{P.roots()[0]}\right\}$

    \Block{else}
    Alors $\Delta = \Var{P.delta} < 0$ donc $P$ n'a pas de racine donc l'équation $\var{P} = 0$ n'a pas de solution.

    \Block{endif}

    \bigskip
    ~\dotfill
    \bigskip
    
    
    \Block{set P = Polynom_deg2.random(["{a}", "{b}", "{c}"])}
    \Block{set Q = Polynom_deg2.random(["{a}", "{b}", "{c}"])}
    Résoudre l'équation suivante
    \begin{eqnarray*}
        \Var{P} & = & \Var{Q}
    \end{eqnarray*}

    Solution:

    On commence par se ramener à une équation de la forme $ax^2+bx+c = 0$.

    \Block{set R = Polynom_deg2((P-Q)._coef)}

    \begin{eqnarray*}
        \Var{P} = \Var{Q} & \Leftrightarrow & \Var{P} - (\Var{Q}) = 0 \\
        \Var{R.explain() | calculus(name = "", sep = "\\Leftrightarrow", end = "= 0")}
    \end{eqnarray*}

    On cherche maintenant à résoudre l'équation $\Var{R} = 0$.
    
    On commence par calculer le discriminant de $R(x) = \Var{R}$.
    \begin{eqnarray*}
        \Delta & = & b^2-4ac \\
        \Var{R.delta.explain()|calculus(name="\\Delta")}
    \end{eqnarray*}
    \Block{set Delta = R.delta}

    \Block{if R.delta > 0}
    comme $\Delta = \Var{R.delta} > 0$ donc $R$ a deux racines

    \begin{eqnarray*}
        x_1 & = & \frac{-b - \sqrt{\Delta}}{2a} =  \frac{\Var{-R.b} - \sqrt{\Var{Delta}}}{2 \times \Var{R.a}} = \Var{R.roots()[0] } \\
        x_2 & = & \frac{-b + \sqrt{\Delta}}{2a} =  \frac{\Var{-R.b} + \sqrt{\Var{Delta}}}{2 \times \Var{R.a}} = \Var{R.roots()[1] }
    \end{eqnarray*}

    Les solutions de l'équation $\Var{R} = 0$ sont donc $\mathcal{S} = \left\{ \Var{min(R.roots())}; \Var{max(R.roots())} \right\}$

    \Block{elif R.delta == 0}
    Comme $\Delta = 0$ donc $R$ a deux racines

    \begin{eqnarray*}
        x_1 = \frac{-b}{2a} = \Var{R.roots()[0]} \\
    \end{eqnarray*}

    La solution de $\Var{R} = 0$ est donc $\mathcal{S} = \left\{ \Var{R.roots()[0]}\right\}$

    \Block{else}
    Alors $\Delta = \Var{R.delta} < 0$ donc $R$ n'a pas de racine donc l'équation $\Var{R} = 0$ n'a pas de solution.

    \Block{endif}

\end{document}

%%% Local Variables: 
%%% mode: latex
%%% TeX-master: "master"
%%% End:

    
