\documentclass[a4paper,10pt]{article}
\RequirePackage[utf8x]{inputenc}
\RequirePackage[francais]{babel}
\RequirePackage{amssymb}
\RequirePackage{amsmath}
\RequirePackage{amsfonts}
\RequirePackage{subfig}
\RequirePackage{graphicx}
\RequirePackage{color}

% Title Page
\title{Calcul littéral et statistiques}
\date{\today}

\begin{document}
\maketitle


\section{Polynômes}


\Block{set P = Polynom.random(["{a}", "{b}", "{c}"], ["{b}**2 - 4*{a}*{c} == 0"])}
    Résoudre l'équation suivante
    \begin{eqnarray*}
        \Var{P} & = & 0
    \end{eqnarray*}

    Solution:

    On commence par calculer le discriminant
    \Block{set Delta = Expression("{b}^2 - 4*{a}*{c}".format(a = P._coef[2], b = P._coef[1], c = P._coef[0]))}
    \begin{eqnarray*}
        \Delta & = & b^2-4ac \\
        \Var{Delta.simplify()|calculus(name="\\Delta")}
    \end{eqnarray*}
    \Block{set Delta = Delta.simplified()}

    \Block{if Delta > 0}
    Alors $\Delta = \Var{Delta} > 0$ donc il y a deux solutions

    \Block{set x1 = (-P._coef[1] - sqrt(Delta))/(2*P._coef[2])}
    \Block{set x2 = (-P._coef[1] + sqrt(Delta))/(2*P._coef[2])}

    \begin{eqnarray*}
        x_1 & = & \frac{-b - \sqrt{\Delta}}{2a} =  \frac{\Var{-P._coef[1]} - \sqrt{\Var{Delta}}}{2 \times \Var{P._coef[2]}} = \Var{x1 | round(2)} \\
        x_2 & = & \frac{-b + \sqrt{\Delta}}{2a} =  \frac{\Var{-P._coef[1]} + \sqrt{\Var{Delta}}}{2 \times \Var{P._coef[2]}} = \Var{x2 | round(2)}
    \end{eqnarray*}

    Les solutions sont donc $\mathcal{S} = \left\{ \Var{x1|round(2)}; \Var{x2|round(2)} \right\}$

    \Block{elif Delta == 0}
    Alors $\Delta = \Var{Delta} = 0$ donc il y a une solution

    \Block{set x1 = Expression("-{b}/(2*{a})".format(b = P._coef[1], a = P._coef[2]))}

    \begin{eqnarray*}
        x_1 = \frac{-b}{2a} = \Var{" = ".join(x1.simplify())}
    \end{eqnarray*}

    Les solutions sont donc $\mathcal{S} = \left\{ \Var{x1.simplified()}\right\}$

    \Block{else}
    Alors $\Delta = \Var{Delta} < 0$ donc il n'y a pas de solution.

    \Block{endif}

    \bigskip
    ~\dotfill
    \bigskip
    
    
    \Block{set P = Polynom.random(["{a}", "{b}", "{c}"])}
    \Block{set Q = Polynom.random(["{a}", "{b}", "{c}"])}
    Résoudre l'équation suivante
    \begin{eqnarray*}
        \Var{P} & = & \Var{Q}
    \end{eqnarray*}

    Solution:

    On commence par se ramener à une équation de la forme $ax^2+bx+c = 0$.

    \begin{eqnarray*}
        \Var{P} = \Var{Q} & \Leftrightarrow & \Var{P} - (\Var{Q}) = 0 \\
        \Var{(P - Q)|calculus(name = "", sep = "\\Leftrightarrow", end = "= 0")}
    \end{eqnarray*}

    \Block{set R = (P-Q)[-1]}
    On cherche maintenant à résoudre l'équation $\Var{R} = 0$.
    
    On commence par calculer le discriminant
    \Block{set Delta = Expression("{b}^2 - 4*{a}*{c}".format(a = R._coef[2], b = R._coef[1], c = R._coef[0]))}
    \begin{eqnarray*}
        \Delta & = & b^2-4ac \\
        \Var{Delta.simplify()|calculus(name="\\Delta")}
    \end{eqnarray*}
    \Block{set Delta = Delta.simplified()}

    \Block{if Delta > 0}
    Alors $\Delta = \Var{Delta} > 0$ donc il y a deux solutions

    \Block{set x1 = (-R._coef[1] - sqrt(Delta))/(2*R._coef[2])}
    \Block{set x2 = (-R._coef[1] + sqrt(Delta))/(2*R._coef[2])}

    \begin{eqnarray*}
        x_1 & = & \frac{-b - \sqrt{\Delta}}{2a} =  \frac{\Var{-R._coef[1]} - \sqrt{\Var{Delta}}}{2 \times \Var{R._coef[2]}} = \Var{x1 | round(2)} \\
        x_2 & = & \frac{-b + \sqrt{\Delta}}{2a} =  \frac{\Var{-R._coef[1]} + \sqrt{\Var{Delta}}}{2 \times \Var{R._coef[2]}} = \Var{x2 | round(2)}
    \end{eqnarray*}

    Les solutions sont donc $\mathcal{S} = \left\{ \Var{x1|round(2)}; \Var{x2|round(2)} \right\}$

    \Block{elif Delta == 0}
    Alors $\Delta = \Var{Delta} = 0$ donc il y a une solution

    \Block{set x1 = Expression("-{b}/(2*{a})".format(b = R._coef[1], a = R._coef[2]))}

    \begin{eqnarray*}
        x_1 = \frac{-b}{2a} = \Var{" = ".join(x1.simplify())}
    \end{eqnarray*}

    Les solutions sont donc $\mathcal{S} = \left\{ \Var{x1.simplified()}\right\}$

    \Block{else}
    Alors $\Delta = \Var{Delta} < 0$ donc il n'y a pas de solution.

    \Block{endif}


\end{document}

%%% Local Variables: 
%%% mode: latex
%%% TeX-master: "master"
%%% End:

    
