\documentclass[a4paper,10pt]{article}
\RequirePackage[utf8x]{inputenc}
\RequirePackage[francais]{babel}
\RequirePackage{amssymb}
\RequirePackage{amsmath}
\RequirePackage{amsfonts}
\RequirePackage{subfig}
\RequirePackage{graphicx}
\RequirePackage{color}



% Title Page
\title{Calcul littéral et statistiques}
\date{\today}

\begin{document}
\maketitle


\section{Polynômes}



    
    Résoudre l'équation suivante
    \begin{eqnarray*}
        3 x^{  2 } + 6 x + 3 & = & 0
    \end{eqnarray*}

    Solution:

    

    On commence par calculer le discriminant de $P(x) = 3 x^{  2 } + 6 x + 3$.
    \begin{eqnarray*}
        \Delta & = & b^2-4ac \\
        \Delta & = & 6^{  2 } - 4 \times 3 \times 3 \\ 
\Delta & = & 36 - 4 \times 9 \\ 
\Delta & = & 36 - 36 \\ 
\Delta & = & 0
    \end{eqnarray*}

    
    Comme $\Delta = 0$ donc $P$ a une racine

    \begin{eqnarray*}
        x_1 = \frac{-b}{2a} = \frac{-6}{2\times 3} = -1 \\
    \end{eqnarray*}

    La solution de $3 x^{  2 } + 6 x + 3 = 0$ est donc $\mathcal{S} = \left\{ -1\right\}$

    



    \bigskip
    ~\dotfill
    \bigskip
    
    
    
    
    Résoudre l'équation suivante
    \begin{eqnarray*}
        x^{  2 } + 4 x + 2 & = & - 9 x^{  2 } + 9 x + 5
    \end{eqnarray*}

    Solution:

    On commence par se ramener à une équation de la forme $ax^2+bx+c = 0$.

    

    \begin{align*}
         & & x^{  2 } + 4 x + 2 = - 9 x^{  2 } + 9 x + 5 \\
         & \Leftrightarrow & x^{  2 } + 4 x + 2 - ( - 9 x^{  2 } + 9 x + 5 )= 0 \\ 
 & \Leftrightarrow & x^{  2 } + 4 x + 2 + 9 x^{  2 } - 9 x - 5= 0 \\ 
 & \Leftrightarrow & ( 1 + 9 ) x^{  2 } + ( 4 - 9 ) x + 2 - 5= 0 \\ 
 & \Leftrightarrow & 10 x^{  2 } - 5 x - 3= 0
    \end{align*}

    On cherche maintenant à résoudre l'équation $10 x^{  2 } - 5 x - 3 = 0$.
    
    

    On commence par calculer le discriminant de $P(x) = 10 x^{  2 } - 5 x - 3$.
    \begin{eqnarray*}
        \Delta & = & b^2-4ac \\
        \Delta & = & -5^{  2 } - 4 \times 10 \times ( -3 ) \\ 
\Delta & = & 25 - 4 \times ( -30 ) \\ 
\Delta & = & 25 - ( -120 ) \\ 
\Delta & = & 145
    \end{eqnarray*}

    
    comme $\Delta = 145 > 0$ donc $P$ a deux racines

    \begin{eqnarray*}
        x_1 & = & \frac{-b - \sqrt{\Delta}}{2a} =  \frac{-5 - \sqrt{145}}{2 \times 10} = - \frac{\sqrt{145}}{20} + \frac{1}{4} \\
        x_2 & = & \frac{-b + \sqrt{\Delta}}{2a} =  \frac{-5 + \sqrt{145}}{2 \times 10} = \frac{1}{4} + \frac{\sqrt{145}}{20}
    \end{eqnarray*}

    Les solutions de l'équation $10 x^{  2 } - 5 x - 3 = 0$ sont donc $\mathcal{S} = \left\{ - \frac{\sqrt{145}}{20} + \frac{1}{4}; \frac{1}{4} + \frac{\sqrt{145}}{20} \right\}$

    



\end{document}

%%% Local Variables: 
%%% mode: latex
%%% TeX-master: "master"
%%% End:

    