\documentclass[a4paper,10pt]{article}
\usepackage[utf8x]{inputenc}
\usepackage[francais]{babel}
\usepackage{amssymb}
\usepackage{amsmath}
\usepackage{amsfonts}

% Title Page
\title{Jouons avec DS\_géné et pyMath}
% \quatreC \quatreD \troisB \troisPro
\date{}


\begin{document}
\maketitle

\section{Exercice de simplification de fraction}
    \Block{do RdExpression.set_form("exp")}
    \Block{set A = RdExpression("{a}/2+2")()}
    \Block{set B = RdExpression("{a}/2+2")()}
		Développer et réduire les expressions suivantes:

		\begin{equation*}
            A = \Var{ A } \qquad
            B = \Var{ B }
		\end{equation*}

        Solutions:
        \Var{A.simplify() | calculus}
        \Var{B.simplify() | calculus(name = "B")}

\section{Mettre sous forme canonique}
    \Block{set P = RdExpression("{a}x^2 + {b}x + {c}")()}
    Mettre $\Var{P}$ sous la forme canonique.

    Solution:

    On simplifie le polynôme:
    \begin{eqnarray*}
        \Var{P.simplify() | calculus(name = "P(x) = ")}
    \end{eqnarray*}
    

    Calcul des coordonnées du sommet de la courbe:
    \begin{eqnarray*}
        \alpha & = & \frac{-b}{2a} =  \\
        \beta & = & -\frac{b^2 - 4ac}{4a} = 
    \end{eqnarray*}
    


\end{document}

%%% Local Variables: 
%%% mode: latex
%%% TeX-master: "master"
%%% End:

