\documentclass[a4paper,10pt]{article}
\RequirePackage[utf8x]{inputenc}
\RequirePackage[francais]{babel}
\RequirePackage{amssymb}
\RequirePackage{amsmath}
\RequirePackage{amsfonts}
\RequirePackage{subfig}
\RequirePackage{graphicx}
\RequirePackage{color}

% Title Page
\title{Calcul littéral et statistiques}
\date{\today}

\begin{document}
\maketitle


\section{Polynômes}



    Résoudre l'équation suivante
    \begin{eqnarray*}
        - 4 x^{  2 } + 4 x - 1 & = & 0
    \end{eqnarray*}

    Solution:

    On commence par calculer le discriminant
    
    \begin{eqnarray*}
        \Delta & = & b^2-4ac \\
        \Delta & = & 4^{  2 } - 4 \times ( -4 ) \times ( -1 ) \\ 
\Delta & = & 16 - ( -16 ) \times ( -1 ) \\ 
\Delta & = & 16 - 16 \\ 
\Delta & = & 0
    \end{eqnarray*}
    

    
    Alors $\Delta = 0 = 0$ donc il y a une solution

    

    \begin{eqnarray*}
        x_1 = \frac{-b}{2a} = \frac{ -4 }{ 2 \times ( -4 ) } = \frac{ -4 }{ -8 } = \frac{ 4 }{ 8 } = \frac{ 1 \times 4 }{ 2 \times 4 } = \frac{ 1 }{ 2 } = \frac{ -4 }{ -8 }
    \end{eqnarray*}

    Les solutions sont donc $\mathcal{S} = \left\{ \frac{ -4 }{ -8 }\right\}$

    

    \bigskip
    ~\dotfill
    \bigskip
    
    
    
    
    Résoudre l'équation suivante
    \begin{eqnarray*}
        - 10 x^{  2 } - 5 x - 5 & = & x^{  2 } + 5 x - 9
    \end{eqnarray*}

    Solution:

    On commence par se ramener à une équation de la forme $ax^2+bx+c = 0$.

    \begin{eqnarray*}
        - 10 x^{  2 } - 5 x - 5 = x^{  2 } + 5 x - 9 & \Leftrightarrow & - 10 x^{  2 } - 5 x - 5 - (x^{  2 } + 5 x - 9) = 0 \\
         & \Leftrightarrow & - 10 x^{  2 } - x^{  2 } - 5 x - 5 x - 5 + 9= 0 \\ 
 & \Leftrightarrow & ( ( -10 ) + ( -1 ) ) x^{  2 } + ( ( -5 ) + ( -5 ) ) x + ( -5 ) + 9= 0 \\ 
 & \Leftrightarrow & - 11 x^{  2 } - 10 x + 4= 0
    \end{eqnarray*}

    
    On cherche maintenant à résoudre l'équation $- 11 x^{  2 } - 10 x + 4 = 0$.
    
    On commence par calculer le discriminant
    
    \begin{eqnarray*}
        \Delta & = & b^2-4ac \\
        \Delta & = & ( -10 )^{  2 } - 4 \times ( -11 ) \times 4 \\ 
\Delta & = & 100 - ( -44 ) \times 4 \\ 
\Delta & = & 100 - ( -176 ) \\ 
\Delta & = & 276
    \end{eqnarray*}
    

    
    Alors $\Delta = 276 > 0$ donc il y a deux solutions

    
    

    \begin{eqnarray*}
        x_1 & = & \frac{-b - \sqrt{\Delta}}{2a} =  \frac{-10 - \sqrt{276}}{2 \times -11} = 0.3 \\
        x_2 & = & \frac{-b + \sqrt{\Delta}}{2a} =  \frac{-10 + \sqrt{276}}{2 \times -11} = -1.21
    \end{eqnarray*}

    Les solutions sont donc $\mathcal{S} = \left\{ 0.3; -1.21 \right\}$

    


\end{document}

%%% Local Variables: 
%%% mode: latex
%%% TeX-master: "master"
%%% End:

    