\documentclass[a4paper,10pt]{article}
\RequirePackage[utf8x]{inputenc}
\RequirePackage[francais]{babel}
\RequirePackage{amssymb}
\RequirePackage{amsmath}
\RequirePackage{amsfonts}
\RequirePackage{subfig}
\RequirePackage{graphicx}
\RequirePackage{color}

% Title Page
\title{Calcul littéral et statistiques}
\date{\today}

\begin{document}
\maketitle


\section{Polynômes}



    Résoudre l'équation suivante
    \begin{eqnarray*}
        - 3 x^{  2 } + 6 x - 3 & = & 0
    \end{eqnarray*}

    Solution:

    On commence par calculer le discriminant
    
    \begin{eqnarray*}
        \Delta & = & b^2-4ac \\
        \Delta & = & 6^{  2 } - 4 \times ( -3 ) \times ( -3 ) \\ 
\Delta & = & 36 - ( -12 ) \times ( -3 ) \\ 
\Delta & = & 36 - 36 \\ 
\Delta & = & 0
    \end{eqnarray*}
    

    
    Alors $\Delta = 0 = 0$ donc il y a une solution

    

    \begin{eqnarray*}
        x_1 = \frac{-b}{2a} = \frac{ -6 }{ 2 \times ( -3 ) } = \frac{ -6 }{ -6 } = \frac{ 6 }{ 6 } = 1 = \frac{ -6 }{ -6 }
    \end{eqnarray*}

    Les solutions sont donc $\mathcal{S} = \left\{ \frac{ -6 }{ -6 }\right\}$

    

    \bigskip
    ~\dotfill
    \bigskip
    
    
    
    
    Résoudre l'équation suivante
    \begin{eqnarray*}
        - 7 x^{  2 } - 7 x + 9 & = & - 2 x^{  2 } + x - 9
    \end{eqnarray*}

    Solution:

    On commence par se ramener à une équation de la forme $ax^2+bx+c = 0$.

    \begin{eqnarray*}
        - 7 x^{  2 } - 7 x + 9 = - 2 x^{  2 } + x - 9 & \Leftrightarrow & - 7 x^{  2 } - 7 x + 9 - (- 2 x^{  2 } + x - 9) = 0 \\
         & \Leftrightarrow & - 7 x^{  2 } + 2 x^{  2 } - 7 x - x + 9 + 9= 0 \\ 
 & \Leftrightarrow & ( ( -7 ) + 2 ) x^{  2 } + ( ( -7 ) + ( -1 ) ) x + 9 + 9= 0 \\ 
 & \Leftrightarrow & - 5 x^{  2 } - 8 x + 18= 0
    \end{eqnarray*}

    
    On cherche maintenant à résoudre l'équation $- 5 x^{  2 } - 8 x + 18 = 0$.
    
    On commence par calculer le discriminant
    
    \begin{eqnarray*}
        \Delta & = & b^2-4ac \\
        \Delta & = & ( -8 )^{  2 } - 4 \times ( -5 ) \times 18 \\ 
\Delta & = & 64 - ( -20 ) \times 18 \\ 
\Delta & = & 64 - ( -360 ) \\ 
\Delta & = & 424
    \end{eqnarray*}
    

    
    Alors $\Delta = 424 > 0$ donc il y a deux solutions

    
    

    \begin{eqnarray*}
        x_1 & = & \frac{-b - \sqrt{\Delta}}{2a} =  \frac{-8 - \sqrt{424}}{2 \times -5} = 1.26 \\
        x_2 & = & \frac{-b + \sqrt{\Delta}}{2a} =  \frac{-8 + \sqrt{424}}{2 \times -5} = -2.86
    \end{eqnarray*}

    Les solutions sont donc $\mathcal{S} = \left\{ 1.26; -2.86 \right\}$

    


\end{document}

%%% Local Variables: 
%%% mode: latex
%%% TeX-master: "master"
%%% End:

    