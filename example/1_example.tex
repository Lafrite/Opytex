\documentclass[a4paper,10pt]{/media/documents/Cours/Prof/Enseignements/Archive/2013-2014/tools/style/classDS}
\usepackage{/media/documents/Cours/Prof/Enseignements/Archive/2013-2014/2013_2014}

% Title Page
\titre{Calcul littéral et statistiques}
% \quatreC \quatreD \troisB \troisPro
\classe{\troisB}
\date{26 septemble 2013}
% DS DSCorr DM DMCorr Corr
\typedoc{DS}
\duree{1 heure}
\sujet{}


\begin{document}
\maketitle

\Calc
Le barème est donné à titre indicatif, il pourra être modifié.

\begin{Exo}[4.5]
    
    
    
    
		Développer et réduire les expressions suivantes:
            \begin{eqnarray*}
                A &=& \frac{ 1 }{ 2 } + 2 \\
                P(x) &=& 6 x - 2 \\
                Q(x) &=& 4 x + 11\\
                R(x) &=& ( 6 x - 2 ) \times ( 4 x + 11 ) 
            \end{eqnarray*}

        Solutions:
        \begin{eqnarray*}
A & = & \frac{ 1 }{ 2 } + 2 \\ 
A & = & \frac{ 1 \times 1 }{ 2 \times 1 } + \frac{ 2 \times 2 }{ 1 \times 2 } \\ 
A & = & \frac{ 1 + 4 }{ 2 } \\ 
A & = & \frac{ 5 }{ 2 }
\end{eqnarray*}

        \begin{eqnarray*}
P(2) & = & 6 \times 2 - 2 \\ 
P(2) & = & 12 - 2 \\ 
P(2) & = & 10
\end{eqnarray*}

        \begin{eqnarray*}
Q(2) & = & 4 \times 2 + 11 \\ 
Q(2) & = & 8 + 11 \\ 
Q(2) & = & 19
\end{eqnarray*}

        \begin{eqnarray*}
P(x) + Q(X) & = & 6 x + 4 x - 2 + 11 \\ 
P(x) + Q(X) & = & ( 6 + 4 ) x + ( -2 ) + 11 \\ 
P(x) + Q(X) & = & 10 x + 9
\end{eqnarray*}

        \begin{eqnarray*}
P(x) + Q(X) & = & 6 x - 2 + 4 x + 11 \\ 
P(x) + Q(X) & = & 4 x + 6 x + 11 - 2 \\ 
P(x) + Q(X) & = & ( 4 + 6 ) x + 11 + ( -2 ) \\ 
P(x) + Q(X) & = & 10 x + 9
\end{eqnarray*}

        \begin{eqnarray*}
R(x) & = & ( 6 x - 2 ) \times ( 4 x + 11 ) \\ 
R(x) & = & 6 \times 4 x^{  2 } + ( -2 ) \times 4 x + 6 \times 11 x + ( -2 ) \times 11 \\ 
R(x) & = & 6 \times 4 x^{  2 } + ( ( -2 ) \times 4 + 6 \times 11 ) x + ( -2 ) \times 11 \\ 
R(x) & = & 24 x^{  2 } + ( ( -8 ) + 66 ) x - 22 \\ 
R(x) & = & 24 x^{  2 } + 58 x - 22
\end{eqnarray*}


\end{Exo}

\begin{Exo}
    
    Résoudre l'équation suivante
    \begin{eqnarray*}
        3 x^{  2 } + x + 10 & = & 0
    \end{eqnarray*}

    Solution:

    On commence par calculer le discriminant
    \begin{eqnarray*}
        \Delta & = & b^2-4ac
    \end{eqnarray*}
    
    \begin{eqnarray*}
\Delta & = & 1^{  2 } - 4 \times 3 \times 10 \\ 
\Delta & = & 1 - 12 \times 10 \\ 
\Delta & = & 1 - 120 \\ 
\Delta & = & -119
\end{eqnarray*}

    
    Alors $\Delta = -119$
    
    
\end{Exo}


\end{document}

%%% Local Variables: 
%%% mode: latex
%%% TeX-master: "master"
%%% End:

    