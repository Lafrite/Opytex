\documentclass[a4paper,10pt]{/media/documents/Cours/Prof/Enseignements/tools/style/classDS}
\usepackage{/media/documents/Cours/Prof/Enseignements/2013_2014}

% Title Page
\titre{Calcul littéral et statistiques}
% \quatreC \quatreD \troisB \troisPro
\classe{\troisB}
\date{26 septemble 2013}
% DS DSCorr DM DMCorr Corr
\typedoc{DS}
\duree{1 heure}
\sujet}}


\begin{document}
\maketitle

\Calc
Le barème est donné à titre indicatif, il pourra être modifié.

\begin{Exo}[4.5]
		Développer et réduire les expressions suivantes:
		\begin{equation*}
				A = %{{random.randint(1,9) %}}x (-%{{random.randint(1,9) %}}x + %{{random.randint(1,9) %}} )\qquad 
				B = (%{{random.randint(1,9) %}}x + %{{random.randint(1,9) %}})(-%{{random.randint(1,9) %}}x + %{{random.randint(1,9) %}}) \qquad 
				C = %{{random.randint(1,9) %}}x (-x + 1) + (%{{random.randint(1,9) %}}x + %{{random.randint(1,9) %}})(x - 4)
		\end{equation*}
\end{Exo}

\begin{Exo}[4]
			On cherche à évaluer l'aire du rectangle suivant:
			\begin{center}
					%\includegraphics[scale=0.3]{fig/rectangle}
			\end{center}
			\begin{enumerate}[a.]
					\item En fonction de $x$, déterminer la hauteur puis la largeur du rectangle.
					\item Exprimer l'aire du rectangle en fonction de $x$.
					\item Réduire l'expression de l'aire.
					\item Évaluer l'aire du rectangle quand $x = %{{random.randint(1,9) %}}$.
			\end{enumerate}
\end{Exo}

\begin{Exo}[6]
		Factoriser, en soulignant le terme commun, les expressions suivantes:
		\begin{equation*}
				A = 6x + 12 \qquad B = 21x + 7 \qquad C = 10x^2 - 5x + 25 \qquad D = 16x^2 - 24x
		\end{equation*}
\end{Exo}

\begin{Exo}[5.5]
		Voici les performances en saut en hauteur d'une classe de troisième. Les hauteurs sont données en centimètres.
		\begin{equation*}
				%{{ "; \\quad ".join(random.gaussRandomlist_strInt(120, 20, 10)) %}}
		\end{equation*}
		\begin{enumerate}[a.]
				\item Quel est l'effectif total de cette série?
				\item Quelle est l'étendue de cette série?
				\item Déterminer la performance moyenne (notée $M$) des élèves de cette classe (On arrondira le résultat à l'unité).
				\item Déterminer une performance médiane (notée $Me$).
		\end{enumerate}

\end{Exo}

\end{document}

%%% Local Variables: 
%%% mode: latex
%%% TeX-master: "master"
%%% End:

